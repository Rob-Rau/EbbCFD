\documentclass[12pt,parskip=full]{article}
\usepackage{lmodern}
\usepackage{amsmath}
\usepackage[left=1.0in,right=1.0in,top=0.5in,bottom=1.0in]{geometry}
\geometry{letterpaper}
\usepackage{graphicx}
\usepackage{caption}
\usepackage{subcaption}
\usepackage{longtable}
\usepackage{float}
\usepackage{wrapfig}
\usepackage{soul}
\usepackage{textcomp}
\usepackage{marvosym}
\usepackage{wasysym}
\usepackage{latexsym}
\usepackage{amssymb}
\usepackage{apacite}
\usepackage{tabu}
\usepackage[svgnames]{xcolor}
\usepackage{tikz}
\usepackage[linktoc=all]{hyperref}
\usepackage{cleveref}
\usepackage{listings}
\usepackage{setspace}
\usepackage{parskip}
\usepackage{array}
\usepackage{apacite}
\usepackage{natbib}
\usepackage{multicol}
\usepackage{subcaption}
\usepackage{mathtools}
\usetikzlibrary{arrows}

\pgfdeclarelayer{edgelayer}
\pgfdeclarelayer{nodelayer}
\pgfsetlayers{edgelayer,nodelayer,main}

\tikzstyle{none}=[inner sep=0pt]
\tikzstyle{waypt}=[circle,fill=Black,draw=Black,scale=0.4]
\tikzstyle{Helobody}=[circle,fill=White,draw=Black,scale=4.0]
\tikzstyle{Tailrotor}=[circle,fill=White,draw=Black,scale=1.0]
\tikzstyle{ForceVector}=[->,draw=Indigo,fill=Indigo]
\tikzstyle{Coordinate}=[->,draw=Red,fill=Red,fill opacity=1.0]
\tikzstyle{angle}=[->]
\tikzstyle{MeasureMark}=[|-|]
\newlength{\imagewidth}
\newlength{\imagescale}

\setlength{\parskip}{11pt}
%\setlength{\parindent}{15pt}
\usepackage{bookmark}
\makeatletter
%\renewcommand\@seccntformat[1]{}
\makeatother

\lstset
{
	language=c,
	keywords={break,case,catch,continue,else,for,
		if,return,switch,try,while,int,void},
	basicstyle=\ttfamily,
	keywordstyle=\color{blue},
	commentstyle=\color{ForestGreen},
	stringstyle=\color{purple},
	numbers=left,
	numberstyle=\tiny\color{gray},
	stepnumber=1,
	numbersep=10pt,
	backgroundcolor=\color{white},
	tabsize=4,
	showspaces=false,
	showstringspaces=false
}

\renewcommand{\thesection}{\arabic{section}}

\renewcommand{\thesubsection}{\thesection\alph{subsection}}
\renewcommand{\theequation}{\thesubsection\arabic{equation}}
\newcommand*\circled[1]{\tikz[baseline=(char.base)]{
			\node[shape=circle,draw,inner sep=1pt] (char) {#1};}}
			
\numberwithin{subsection}{section}

\begin{document}
	\vspace{-4ex}
	\title{EbbCFD Theory Guide Part 2\vspace{-3.5ex}}
	\author{Rob Rau\vspace{-4ex}}
	\date{\today\vspace{-4ex}}
	\maketitle

	\section{Introduction}
		In the last three months I have developed EbbCFD to incorporate the beginnings of a Navier-Stokes finite volume solver.
		Additionally to aid in future development, the method of Manufactured solutions has been implemented. EbbCFD has now been
		more thoroughly tested with quadrilateral elements and mixed element meshes.

	\section{Navier-Stokes}
		The current Navier-Stokes implementation is a rather naive approach. It modifies the convective flux terms by adding a
		viscous flux term. The viscous term is computed as
		\begin{equation}
		\end{equation}
		This involves making a approximation of the solution gradient at the cell interface. EbbCFD currently does this by simply
		taking the average of the solution gradient of the neighboring cells.
		
		
		This provides a stable scheme at low Reynolds numbers, but does not show favorable convergence, as shown later. To address convergence I plan
		on implementing a version of the Bassi-Rebay 2 viscous discritization adapted for finite volumes.

	\section{Method of Manufactured Solutions}
		To show that EbbCFD had correctly implimented its discritization, I employed the Method of Manufactured Solutions.
		The exact solution chosen was
		\begin{align}
			\rho &= a_\rho + b_\rho \sin(c_\rho x + d_\rho y) \\
			u &= a_u + b_u \cos(c_u x + d_u y) \\
			v &= a_v + b_v \cos(c_v x + d_v y) \\
			p &= a_p + b_p \sin(c_p x + d_p y)
		\end{align}

\end{document}
