\documentclass[12pt,parskip=full]{article}
\usepackage{lmodern}
\usepackage{amsmath}
\usepackage[left=1.0in,right=1.0in,top=0.5in,bottom=1.0in]{geometry}
\geometry{letterpaper}
\usepackage{graphicx}
\usepackage{caption}
\usepackage{subcaption}
\usepackage{longtable}
\usepackage{float}
\usepackage{wrapfig}
\usepackage{soul}
\usepackage{textcomp}
\usepackage{marvosym}
\usepackage{wasysym}
\usepackage{latexsym}
\usepackage{amssymb}
\usepackage{apacite}
\usepackage{tabu}
\usepackage[svgnames]{xcolor}
\usepackage{tikz}
\usepackage[linktoc=all]{hyperref}
\usepackage{cleveref}
\usepackage{listings}
\usepackage{setspace}
\usepackage{parskip}
\usepackage{array}
\usepackage{apacite}
\usepackage{natbib}
\usepackage{multicol}
\usepackage{subcaption}
\usepackage{mathtools}
\usetikzlibrary{arrows}

\pgfdeclarelayer{edgelayer}
\pgfdeclarelayer{nodelayer}
\pgfsetlayers{edgelayer,nodelayer,main}

\tikzstyle{none}=[inner sep=0pt]
\tikzstyle{waypt}=[circle,fill=Black,draw=Black,scale=0.4]
\tikzstyle{Helobody}=[circle,fill=White,draw=Black,scale=4.0]
\tikzstyle{Tailrotor}=[circle,fill=White,draw=Black,scale=1.0]
\tikzstyle{ForceVector}=[->,draw=Indigo,fill=Indigo]
\tikzstyle{Coordinate}=[->,draw=Red,fill=Red,fill opacity=1.0]
\tikzstyle{angle}=[->]
\tikzstyle{MeasureMark}=[|-|]
\newlength{\imagewidth}
\newlength{\imagescale}

\setlength{\parskip}{11pt}
%\setlength{\parindent}{15pt}
\usepackage{bookmark}
\makeatletter
%\renewcommand\@seccntformat[1]{}
\makeatother

\lstset
{
	language=c,
	keywords={break,case,catch,continue,else,for,
		if,return,switch,try,while,int,void},
	basicstyle=\ttfamily,
	keywordstyle=\color{blue},
	commentstyle=\color{ForestGreen},
	stringstyle=\color{purple},
	numbers=left,
	numberstyle=\tiny\color{gray},
	stepnumber=1,
	numbersep=10pt,
	backgroundcolor=\color{white},
	tabsize=4,
	showspaces=false,
	showstringspaces=false
}

\renewcommand{\thesection}{\arabic{section}}

\renewcommand{\thesubsection}{\thesection\alph{subsection}}
\renewcommand{\theequation}{\thesubsection\arabic{equation}}
\newcommand*\circled[1]{\tikz[baseline=(char.base)]{
			\node[shape=circle,draw,inner sep=1pt] (char) {#1};}}
			
\numberwithin{subsection}{section}

\begin{document}
	\vspace{-4ex}
	\title{EbbCFD Theory Guide\vspace{-3.5ex}}
	\author{Rob Rau\vspace{-4ex}}
	\date{\today\vspace{-4ex}}
	\maketitle

	\section{Introduction}
		EbbCFD is a computational fluid dynamics package intended to solve the compressible Euler equations.
		It uses an unstructured, limited, second order finite volume spacial scheme and supports numerous
		time integration schemes. This document covers the theory behind EbbCFD's implimentation

	\section{Unstructured Finite Volume}
		EbbCFD operates on completely unstructured meshes. In theory it supports cell geometries of up to six faces,
		though only trianglular meshes have been extensivly tested. For a first order spatial update, the numerical flux
		is computed on each edge in the mesh, than, for each cell in the mesh, those fluxes are summed and called the
		flux residuals.
		\begin{equation}
			\mathbf{R}_i = \sum_{e = 1}^{N_{f_i}}{m\mathbf{\hat{F}}(\mathbf{U}_i, \mathbf{U}_{N(i,e)}, \vec{n}_{i,e}) \Delta l_{i,e}}
		\end{equation}
		where $\mathbf{\hat{F}}(\mathbf{U}_i, \mathbf{U}_{N(i,e)}, \vec{n}_{i,e})$ is the numerical flux function,
		$\Delta l_{i,e}$ is the length of the cell edge, $N_{f_i}$ is the number of faces of cell $i$, and $m$ is $\pm 1$
		depending on which cell the normal $\vec{n}_{i,e}$ belonged to. This possible because the flux function is homogeneous, i.e.
		$\mathbf{\hat{F}}(\mathbf{U}_i, \mathbf{U}_{N(i,e)}, \vec{n}_{i,e}) = -\mathbf{\hat{F}}(\mathbf{U}_i, \mathbf{U}_{N(i,e)}, -\vec{n}_{i,e})$.
		This allows the code to compute the flux only once per edge, and then depending on the cell, multiply by $\pm 1$.
		With this flux residual we can write the semi-discrete form of the first order finite volume update.
		\begin{equation}
			\frac{\partial \mathbf{U}_i}{\partial t} = -\frac{1}{A_i}\mathbf{R}_i
		\end{equation}
		where $A_i$ is the area of cell $i$. Currently, EbbCFD only supports the Roe flux function, but others are planned.

	\section{Second Order}
		To acheive second order accuracy, the numerical flux is no longer computed using cell averaged state values.
		Instead, the edge state values are computed and used. To obtain edge state values, an estimate of the cell gradients 
		is necessary. To reduce solution oscilations at discontinuities in the flow, EbbCFD also supports a couple of
		different schemes for gradient limiting.

		\subsection{Gradients}
			To estimate cell gradients a least squares approximation was used. The least squares problem is set up like so
			\begin{equation}
				\mathbf{r} = \begin{bmatrix} x_1 - x_M & y_1 - y_M \\ \vdots & \vdots \\ x_{N_{f_i}} - x_M & y_{N_{f_i}} - y_M \end{bmatrix} \begin{bmatrix}D_x \\ D_y \end{bmatrix} - \begin{bmatrix} u_1 - u_M \\ \vdots \\ u_{N_{f_i}} - u_M \end{bmatrix}
			\end{equation}
			where $x_M$ and $y_M$ are the current cell centroid coordinates, $x_1..x_{N_{f_i}}$ and $y_1..y_{N_{f_i}}$ 
			are the cell centroid coordinates for the neighboring cells, $u_M$ is a conserved state variable of the current 
			cell, and $u_1..u_{N_{f_i}}$ are the conserved state variables of the neighboring cells. In 2D, four
			least squares problems need to be solved in order to get a gradient for each conserved quantity (density,
			x/y momentum, and energy for the Euler equations). For cells located at a boundary, a ghost cell approach is
			used. The cell centroid of the boundary cells is reflected accross the boundary edge and its state values
			are updated in acordance with what type of boundary it is. A wall boundary for instance, will reflect the
			flow momentum accross the boundary in order to cancel out the normal momentum component.

		\subsection{Limiters}
			EbbCFD currently supports two different limiting techniques. The first is a scalar limiter, where a single scalar
			value limits both x and y components of a gradient. The other is a vector limiter, where x and y gradient components
			are limited seperatly. 
			
	\section{Numerical Viscosity}


\end{document}
